\documentclass{article}
\usepackage{cv}
\usepackage{graphicx}

\begin{document}

\begin{center}
\textsc{\small phy 4910U Techniques of Modern Astrophysics | Winter 2025 }

\vspace{0.1in}  

\textsc{ {\Large Worksheet 1-1-2  |  Working With Data }}
\normalsize
\vspace{0.2in}
\end{center}



\cvsection{A.  Generate Some Data}

First we need to create some fake data to work with.  We want a single file, called \texttt{data.txt}, which has two columns of numbers -- the first column $x_i$ has 100 numbers going from 0 to 1.0, and the second column $f_i$ corresponds to the function
\[
f(x) = xe^{-x^2}.
\]

Use Python, with NumPy for the arrays and output, to create this file.  Be sure to create a new directory in your repository to work in.

\cvsection{B.  Plot Some Data}

Now plot the data, using these basic steps:

\begin{enumerate}
\item Create a Python file called \texttt{plot.py}.
\item Use NumPy to read in the text file and assign arrays to each column.
\item Use MatPlotLib to plot the data and display it on the screen.
\item Make the Python file executable so you can run it from the terminal: \texttt{\$ ./plot.py data.txt}.
\end{enumerate}

Finally, consider some extensions to what you have:
\begin{itemize}
\item Comment the file so it's readable to others.
\item Include an option to save the plot as a PDF file or PNG image.
\item Extend the code to allow more than two columns, and provide an option for which columns to plot.
\item Include an option to set the $x$ and $y$ limits of the plot.
\end{itemize}
Plus whatever else you think will be useful later.

\cvsection{C.  GitHub}

After showing me that your \verb|plot.py| file works according to the minimum specifications above, push it to your GitHub repository.  A working \verb|plot.py| file in the repository will be graded for completion. 


\end{document}
